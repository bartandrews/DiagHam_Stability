\documentclass[aps,prb,twocolumn,letterpaper,twoside,nobalancelastpage,groupedaddress,amsmath,amssymb,floatfix,citeautoscript]{revtex4-1}
%\usepackage{geometry}       
%\geometry{letterpaper}     
%\usepackage[parfill]{parskip}    % Activate to begin paragraphs with an empty line rather than an indent
\usepackage{graphicx}
\usepackage{bm} %bold math symbols
\usepackage{times}
\usepackage{graphicx}
\usepackage{physics}
\usepackage{outlines}             
\usepackage{amssymb}
\usepackage[caption=false]{subfig}
\usepackage{mathtools}
\usepackage{color}
\definecolor{darkred}{rgb}{0.6,0.,0.}
\definecolor{darkgreen}{rgb}{0.,0.5,0.}
\definecolor{darkblue}{rgb}{0.,0.,0.6}

\usepackage{framed}

%\usepackage[square,numbers,sort,merge]{natbib}
\usepackage[
breaklinks,
colorlinks=true,
linkcolor=darkred,
citecolor=darkgreen,
urlcolor=darkblue]{hyperref}
%\usepackage{doi}

% \DeclareMathOperator{\tr}{tr}
% \DeclareMathOperator{\erf}{Erf}

\DeclareGraphicsRule{.tif}{png}{.png}{`convert #1 `dirname #1`/`basename #1 .tif`.png}

\begin{document}


\def \Ns {\mathbb{N}}
\def \Rs {\mathbb{R}}
\def \Zs {\mathbb{Z}}
\def \Qs {\mathbb{Q}}
\def \Cs {\mathbb{C}}
\def \id {\mathbb{I}}

\def \bfq {{\bf q}}
\def \bfp {{\bf p}}
\def \bfx {{\bf x}}
\def \bfy {{\bf y}}
\def \bfz {{\bf z}}
\def \bfr {{\bf r}}
\def \bfk {{\bf k}}
\def \bfn {{\bf n}}
\def \bfb {{\bf b}}
\def \bfm {\mathbf{m}}
\def \bfn {\mathbf{n}}

%\def \hata {{a}}
%\def \hatadag {{a}^{\dagger}}

\def \hatq {\widehat{q}}
\def \hatp {\widehat{p}}
\def \hata {\widehat{a}}
\def \hatadag {\widehat{a}^{\dagger}}
%\def \hatb {\widehat{b}^\phantom{\dagger}}
%\def \hatbdag {\widehat{b}^{\dagger}}
\def \wtN {\widetilde{N}}

\def \ve {\varepsilon}
\def \vth {\vartheta}



\title{Lattice models for generic Landau orbits}
\author{David Bauer}
\email{dbauer@physics.ucla.edu}
\affiliation{Department of Physics and Astronomy, University of California at Los Angeles, 475 Portola Plaza, Los Angeles, California 90095, USA}

\author{Fenner Harper}
\affiliation{Department of Physics and Astronomy, University of California at Los Angeles, 475 Portola Plaza, Los Angeles, California 90095, USA}

\author{Rahul Roy}
\affiliation{Department of Physics and Astronomy, University of California at Los Angeles, 475 Portola Plaza, Los Angeles, California 90095, USA}

\date{\today}
\begin{abstract}
Abstract goes here.
% Recent work on the quantum Hall effect (QHE) in lattice models and other systems with non-trivial geometry has opened the question of the extent to the stability of quantum Hall states is dependent on Landau levels. We construct a lattice model that has as its continuum limit a Hamiltonian with purely quartic momentum dependence, yielding single-particle states different from Landau levels in which one may study the QHE. We study the spectrum and band structure of the lattice Hamiltonian and its continuum limit, as well as the fractional quantum Hall effect (FQHE) of interacting particles in these bands. 
\end{abstract}

% \pacs{0}

\maketitle
% \begin{outline}
% \1 Introduction
% \1 Effective continuum hamiltonians
% \1 The quartic model -- spectrum, band geometry, etc.
% \1 FQH physics in the quartic LLL
% \1 Discussion
% \end{outline}
\section{Introduction}
\subsection{Background and motivation}
The incompressible liquid phases of the fractional quantum Hall (FQH) effect serve as prototypical examples of topologically ordered or gapped quantum liquid phases.

Two recent developments in FQH physics have each inspired a large body of literature. The first is the numerical discovery of fractional Chern insulators (FCI), which are fractionalized phases of interacting particles in the presence of a strong lattice potential. This is in contrast to the traditional picture of the quantum Hall effect, which is formulated in the continuum. Since FQH phases are topologically stable against local perturbations,

FCI phases have been discovered experimentally in bilayer graphene heterostructures [cite: Spanton et al.] The single-particle Chern bands that are the FCI analogues of Landau levels have also been observed experimentally in cold atoms [cite].

A second development in the FQH literature is based on observations that, in spite of their ``topological'' nature, FQH liquids are also characterized by non-trivial \textit{geometrical} properties.

As pointed out by Haldane [cite], the geometry inherent in the FQH problem has been obscured by the introduction of symmetries -- for example rotational invariance -- that are unnecessary for the stability of the FQH liquid. 

These two threads of inquiry are not disconnected. The addition of a non-negligible lattice potential introduces additional geometric data that may break some non-generic symmetries. For example, in the case of a square lattice, $SO(2)$ rotational invariance in the coordinate plane is explicitly broken to $D_4$ lattice symmetry. From this point of view, a complete theory of the FQHE formulated as generically as possible should furnish a theory of FCI phases, or at least those FCI phases that have FQH analogoues. [It is not clear to me whether there are topologically-ordered phases in FCIs that require going beyond this generic FQH framework.]


Although the FQHE is essentially a many-particle phenomenon, we expect that the nature of the host band should play a crucial role. 


In the following section, we show that under some fairly generic assumptions, the effective hamiltonian will be quadratic in the momenta. In other words, Landau levels are a generic.


\subsection{Universality of Landau orbits}
In this section, we consider a tight-binding model of an electron living on a Bravais lattice in two spatial dimensions. 

We will assume that the hopping amplitude between any two sites on the lattice is generically non-zero. We index the sites of the lattice by a vector $\mathbf{m} = (m_1, m_2)$ with integer components. Let $c^{\dag}_{\mathbf{m}}$ ($c_{\mathbf{m}}$) create (annihilate) an electron at site $\mathbf{m}$ of the lattice. We have the usual fermion anticommutation relations $\left\{c_{\mathbf{m}},c_{\mathbf{n}}^{\dag}\right\} = 2\delta_{\mathbf{m} \mathbf{n}}$. The single-particle Hilbert space is spanned by the space of states $\ket{\mathbf{m}}$ with the electron occupying site $\mathbf{m}$. The hamiltonian is
\begin{align*}
H &= \sum_{\mathbf{m}\neq \mathbf{n}} t_{\mathbf{m}\mathbf{n}} c^{\dag}_\mathbf{n} c_\mathbf{m}  + t_{\mathbf{n}\mathbf{m}} c^{\dag}_{\mathbf{m}} c_{\mathbf{n}}\\  &= \sum_{\mathbf{m}\neq \mathbf{n}} t_{\mathbf{m}\mathbf{n}} \dyad{\mathbf{n}}{\mathbf{m}} + t_{\mathbf{n}\mathbf{m}} \dyad{\mathbf{m}}{\mathbf{n}}
\end{align*}
Note that we are excluding onsite/mass terms from the above hamiltonian; this is because a translation-invariant mass term simply shifts $H$ by a constant. We define lattice translation operators $T_a = \sum_{\bfm} \dyad{\mathbf{m} + \mathbf{e}_a}{\mathbf{m}}$. We can write the hamiltonian in terms of these as
\begin{align}
\label{eq-b0-lattice-hamiltonian}
H = \sum_{j,k} t_{jk} \left(T_1^j T_2^k + (T^{\dag}_2)^{k} (T^{\dag}_1)^{j}\right)
\end{align}

We introduce a uniform background magnetic field $B$ perpendicular to the spatial extent of the lattice. We choose the value of $B$ to be such that the flux per lattice plaquette is $Ba^2 = \frac{P}{Q}\Phi_0$, where $\Phi_0 = 2\pi \hbar /e$ is the magnetic flux quantum and $P$ and $Q$ are relatively prime integers. In terms of the magnetic length $\ell = \sqrt{\frac{\hbar}{eB}}$, we have $a^2/\ell^2 = 2 \pi P/Q$. We now define $\epsilon \coloneqq a^2/\ell^2$. In the presence of the magnetic field, the above translation operators are no longer gauge invariant. Instead, the correct lattice translation operators in this case are
\begin{align*}
\widetilde{T}_a = \sum_{\mathbf{m}} e^{i\theta_a(\mathbf{m})} \dyad{\bfm + \mathbf{e}_a}{\bfm}.
\end{align*}
where the phases $e^{i\theta_a(\bfm)}$ satisfy 

\begin{align*}
\theta_1(\bfm) + \theta_2(\bfm + \mathbf{e}_1) - \theta_1(\bfm + \mathbf{e}_2) - \theta_2(\bfm) = \epsilon.
\end{align*}

The lattice translation operators $\widetilde{T}_a$ are unitary, so we can write them in terms of hermitian generators $\widetilde{T}_a = \exp\left[-i \Pi_a\right]$. The components of $\widetilde{\mathbf{T}}$ do not commute, but satisfy $\widetilde{T}_1 \widetilde{T}_2 = \exp(i\epsilon) \widetilde{T}_2 \widetilde{T}_1 $. This implies the commutator
\begin{align*}
\comm{\Pi_1}{\Pi_2} = i \epsilon.
\end{align*}
[Actually, it seems like it implies $\comm{\Pi_1}{\Pi_2} = i \epsilon + 2\pi i M$ for $M \in \mathbf{Z}$. I think it's probably worthwhile to keep track of the $M$, although I'm not sure of its significance. (Also I guess for the same reason the phases should really satisfy $\sum_{\square}\theta = \epsilon + 2\pi M$). Anyway, I think I'm ok to ignore this here.]

Because the $\widetilde{T}_a$ do not commute with each other, there is an ambiguity in passing from the $B=0$ hamiltonian (\ref{eq-b0-lattice-hamiltonian}) to one for $B\neq0$, which appears as a choice of phase of the (now-complex) $t_{jk}$. This is analogous to the ordering ambiguity in quantizing polynomials on classical phase space. We fix this ambiguity by making the arbitrary choice that $T_1^j T_2^k \rightarrow \widetilde{T}_1^j \widetilde{T}_2^k + \widetilde{T}_2^k\widetilde{T}_1^j$ in the presence of nonzero $B$. Then
\begin{align*}
H = \sum_{j,k} t_{jk}\left(\widetilde{T}_1^j \widetilde{T}_2^k + \widetilde{T}_2^k\widetilde{T}_1^j\right) + \text{h.c.}
\end{align*}


Let's look at just the local part of this hamiltonian, $H_{\text{loc}} = t_{10}T_1 + t^{\ast}_{10}T_1^{\dag} + t_{01}T_2 + t^{\ast}_{01}T_2^{\dag}$

If we set $t_{10} = t_{01}$, $H_{\text{loc}}$ is just the hamiltonian of the Hofstadter model





Now let $\theta_a = \epsilon \chi_a$, so that
\begin{align*}
\chi_1(\bfm) + \chi_2(\bfm + \mathbf{e}_1) - \chi_1(\bfm + \mathbf{e}_2) - \chi_2(\bfm) = 2\pi
\end{align*}


\subsection{Chern bands and geometry}
By a Chern band, we will mean

Berry curvature

Fubini-Study metric


\subsection{Relation to previous work}
[lattice FQH, FCIs]
[Haldane]

\subsection{Outline and summary}
In the following section, we will consider particular lattice models for which the continuum-limit hamiltonian contains both quadratic and quartic terms. We will 

Then we will specialize to the case in which the quadratic term vanishes identically. 

Finally, we will comment on anisotropic modifications to these models when the lattice symmetry is broken further by varying the relative hopping strengths in each lattice direction.

\section{Generic quartic model}
Following the above discussion, the most general effective hamiltonian we can write down, keeping terms quartic in the momenta, is 
\begin{align*}
H = \frac{1}{2}h^{(1)}_{ab} \pi_a \pi_b + \frac{1}{4}h^{(2)}_{abcd} \pi_a \pi_b \pi_c \pi_d
\end{align*}
As discussed, there is ambiguity in ordering the components of the momentum. We will for now elide this ambiguity by a particular choice of ordering. Namely, we choose the coefficients $h^{(n)}$ to be totally symmetric in their $2n$ indices.

Under an $SL(2,R)$ transformation of the momenta, the commutator $\comm{\pi}{\pi}$ is preserved.

In general, there are no restrictions on the form of $h^{(2)}_{abcd}$ beyond total symmetry of its indices. However, in the next section we will consider the case in which $h^{(1)}$ vanishes. In this case, $H$ will not have a stable minimum unless $h^{(2)}_{abcd} = \gamma^1_{ab}\gamma^2_{cd}$ with $\gamma^j_{ab} = \gamma^{j}_{ba}$ symmetric.


\section{Purely quartic model}
Now we consider the continuum limit of the following lattice model.




\section{Anistropic models}

\begin{acknowledgments}

\end{acknowledgments}

\bibliography{quartic-2}
% \bibliography{hofstadter-manual}

\end{document}


